\documentclass{article}
\usepackage{babel}
\usepackage[a4paper, top = 20mm, bottom=20mm, right=20mm, left=20mm] {geomet    ry}
\usepackage{graphicx}
\usepackage{amsmath}
\usepackage{wrapfig}
\usepackage{xcolor}
\usepackage{matlab-prettifier}
\usepackage[hidelinks]{hyperref}
\setlength{\parindent}{3mm}
\setlength{\parskip}{5mm}
\linespread{1.65}
\renewcommand{\figurename}{Fig.}

\title{\textbf{Práctica 2 Análisis de Señales}}
\date{\today}
\author{
\href{mailto:igamher@etsid.upv.es}{Ignacio Amat Hernández}
\thanks{\href{https://www.upv.es/titulaciones/GIB/indexc.html}{Grado en Ingeniería Biomédica, Escuela Técnica Superior de Ingenieros Industriales, Valencia, España.}}
}


\begin{document}
\maketitle
\section{Ejercicio 1 Fast Fourier Transform vs Periodograma (Welch)}

Primero dibujamos la \textit{Fast Fourier Transform} de la señal:
\vfill
\begin{figure}[h]
\centering
\includegraphics[width = \linewidth]{../plots/fft.pdf}
\vspace{-1.5cm}
\caption{\textit{Fast Fourier Transform} del paciente \lstinline[style=matlab-editor]{001cltv1}.}
\label{fig:fft}
\end{figure}
\vfill
\newpage

\subsection{¿Cuál es la frecuencia del pico principal?}
Ahora dibujamos en detalle los primeros 20$Hz$, marcamos en rojo los
picos y la escribimos la frecuencia a la que ocurren.
\begin{equation}
f_{PP} = 2.85 Hz
\end{equation}
\vspace{-1cm}
\begin{figure}[h]
\centering
\includegraphics[width = \linewidth]{../plots/fft1.pdf}
\vspace{-1.5cm}
\caption{Detalle de los primeros 20$Hz$ de la \textit{FFT}.}
\label{fig:picos}
\end{figure}

\vspace{-1cm}
\subsection{¿Cuál es el ancho del pico principal?}

Calculamos los picos y las anchuras con la función
\lstinline[style=matlab-editor]{findpeaks} de
\textsc{matlab}.
El pico principal tiene una anchura a media altura (\textit{FWHM}) de
$0.2Hz$ como se muestra en la \hyperref[fig:picos]{\textbf{Fig.}~\ref*{fig:picos}}.
\begin{equation}
FWHM = 0.20 Hz
\end{equation}
\vspace{-1.5cm}
\subsection{¿Tiene  armónicos?  ¿Cuántos? ¿Cómo es la amplitud  de
los armónicos con respecto al pico principal?}

Para investigar los armónicos primero tomamos el vector con las
frecuencias a las que ocurren los picos de la
\hyperref[fig:picos]{\textbf{Fig.}~\ref*{fig:picos}} y dividimos cada
entrada por el valor del segundo pico (el pico principal). Obtenemos
estos resultados:

\begin{table}[h!]
\centering
\begin{tabular}{|r|c|c|c|c|c|c|c|c|c|}
\hline
\textbf{Picos} & \textbf{1} & \textbf{2} & \textbf{3} & \textbf{4} &
	\textbf{5} & \textbf{6} & \textbf{7} & \textbf{8}\\\hline
Frecuencias & 0.47 & 2.85 & 5.94 & 8.43 & 11.28 & 14.01 & 17.57 & 19.59\\\hline
Normalizadas & 0.17 & 1.00 & 2.08 & 2.96 & 3.96 & 4.92 & 6.17 & 6.88\\\hline
Redondeadas  & 0 & 1 & 2 & 3 & 4 & 5 & 6 & 7\\\hline
\end{tabular}
\caption{Frecuencias de los picos.}
\label{table:fcs}
\end{table}
\newpage
En la \hyperref[table:fcs]{\textbf{Tbl.}~\ref*{table:fcs}} vemos que
cuando redondeamos las frecuencias normalizadas al pico principal
obtenemos una secuencia perfecta de números del 1 al 7; esto indica
que los picos se corresponden con los armónicos del segundo pico.
Encontramos que el pico 2 tiene 6 armónicos superiores en los primeros
20Hz de señal, es de esperar que tenga más, pero su amplitud es
demasiado pequeña para poder ser detectados. Las amplitudes se
muestran en al \hyperref[fig:picos]{\textbf{Fig.}~\ref*{fig:picos}}.

\subsection{¿Cuál es la resolución en frecuencia, es decir, el paso entre un punto y otro?}

La resolución en frecuencia indica a partir de qué frecuencia vamos a
ver la señal, y también cada cuánto se van a tomar las muestras.  La
resolución en frecuencia viene dada por el cociente entre la
frecuencia de muestreo y el número de puntos de la transformada de
Fourier, en este caso:
\begin{equation}
f_{r} = \dfrac{250 Hz}{2107} = 0.118652 Hz
\end{equation}

\subsection{¿De  qué  depende  la  resolución  en  frecuencia? ¿Qué  se  podría  hacer  para aumentar el número de puntos, y así aumentar la resolución?}

Vemos que la resolución en frecuencia depende del tramo que escojamos
para el análisis y del envenenado, ya que son las frecuencias que dan
más señal. Si aumentamos el intervalo de muestras o reducimos la
frecuencia de muestreo, la resolución mejora. Si es un tramo demasiado
pequeño no la veremos correctamente, y si es demasiado grande tampoco.
La resolución de compromiso, para estos casos en particular, unos 8
seg puede ser óptimo (aunque también se pueden escoger tramos de  4,
16 segundos...).

\vfill
\begin{figure}[h]
\centering
\includegraphics[width = \linewidth]{../plots/fft2.pdf}
\vspace{-1.5cm}
\caption{Variabilidad del espectro \textit{FFT}.}
\label{fig:LABEL_NAME}
\end{figure}
\vfill
\newpage

\subsection{Calcule la densidad espectral de potencia}

\begin{figure}[h]
\centering
\includegraphics[width = \linewidth]{../plots/welch1.pdf}
\vspace{-1cm}
\caption{Periodograma de \textit{Welch}.}
\label{fig:welch}
\end{figure}


\vfill
\subsection{¿Sobre cuántos tramos se realiza el promediado?}

La señal entera tiene 2107 muestras y nosotros usamos una ventana de
$4 \text{ secs} * f_{s} = 1000$ muestras con un solape de 500 muestras.
El promedio se realizará sobre tres tramos.

\begin{figure}[h]
\centering
\includegraphics[width = \linewidth]{../plots/welch2.pdf}
\vspace{-1cm}
\caption{Periodograma de \textit{Welch}.}
\label{fig:welch}
\end{figure}

\subsection{¿Cómo cambia el ancho del pulso? ¿Por qué?}

La densidad espectral de potencia (PSD) se define como la variación de
energía que hay dentro de una señal vibratoria, en función de la
frecuencia por unidad de masa. Es decir, muestra para cada frecuencia
si la energía presente es mayor o menor.

El periodograma es un estimador de la densidad espectral de potencia
que permite realizar un suavizado del espectro. Con ello obtenemos una
señal más estética pero perdemos resolución. Podemos además escoger
ventanas y realizar solapes entre ellas y así obtener un mayor
suavizado. Se selecciona un segmento, se reduce el tamaño de las
ventanas, se promedia. Además podemos escoger el solape que nos
convenga.

\subsection{¿Qué espectro es más fiable?}

El espectro proporcionado por el periodograma de \textit{Welch} es
menos fiable que el proporcionado por la transformada de
\textit{Fourier}, esto es debido a que el periodograma suaviza la
señal. Este suavizado inherente a la técnica implica
perder parte de la información.

\newpage
\begin{figure}[h]
\centering
\includegraphics[width = \linewidth]{../plots/2welch1.pdf}
\includegraphics[width = \linewidth]{../plots/2welch2.pdf}
\includegraphics[width = \linewidth]{../plots/2welch3.pdf}
\end{figure}

\begin{figure}[h]
\centering
\includegraphics[width = \linewidth]{../plots/2welch4.pdf}
\includegraphics[width = \linewidth]{../plots/2welch5.pdf}
\includegraphics[width = \linewidth]{../plots/2welch6.pdf}
\end{figure}



\newpage

\begin{figure}[h]
\centering
\includegraphics[width = \linewidth]{../plots/3welch1.pdf}
\includegraphics[width = \linewidth]{../plots/3welch2.pdf}
\includegraphics[width = \linewidth]{../plots/3welch3.pdf}
\end{figure}

\begin{figure}[h]
\centering
\includegraphics[width = \linewidth]{../plots/3welch4.pdf}
\includegraphics[width = \linewidth]{../plots/3welch5.pdf}
\includegraphics[width = \linewidth]{../plots/3welch6.pdf}
\end{figure}

\begin{figure}[h]
\centering
\includegraphics[width = \linewidth]{../plots/3welch7.pdf}
\includegraphics[width = \linewidth]{../plots/3welch8.pdf}
\includegraphics[width = \linewidth]{../plots/3welch9.pdf}
\end{figure}

\begin{figure}[h]
\centering
\includegraphics[width = \linewidth]{../plots/3welch10.pdf}
\end{figure}

\end{document}
