\documentclass{article}
\usepackage{babel}
\usepackage[a4paper, top = 20mm, bottom=20mm, right=20mm, left=20mm] {geomet    ry}
\usepackage{graphicx}
\usepackage{amsmath}
\usepackage{wrapfig}
\usepackage{xcolor}
\usepackage{matlab-prettifier}
\usepackage[hidelinks]{hyperref}
\setlength{\parindent}{3mm}
\setlength{\parskip}{5mm}
\linespread{1.65}
\renewcommand{\figurename}{Fig.}

\title{\textbf{Práctica 2 Análisis de Señales}\thanks{\href{https://www.upv.es/titulaciones/GIB/indexc.html}{Grado en Ingeniería Biomédica, Escuela Técnica Superior de Ingenieros Industriales, Valencia, España.}}}

\date{\today}
\author{
     \href{mailto:igamher@etsid.upv.es}{Ignacio Amat Hernández}
\and \href{mailto:ngellohe@etsii.upv.es}{Ángela LLopis Hernández}
\and \href{mailto:esbalho@etsii.upv.es}{Estrella Ballester Hoyos}
}

\begin{document}
\maketitle
\section{Ejercicio 1 Fast Fourier Transform vs Periodograma (Welch)}

Primero dibujamos la \textit{Fast Fourier Transform} de la señal:
\vfill
\begin{figure}[h]
\centering
\includegraphics[width = \linewidth]{../plots/fft.pdf}
\vspace{-1.5cm}
\caption{\textit{Fast Fourier Transform} del paciente \lstinline[style=matlab-editor]{001cltv1}.}
\label{fig:fft}
\end{figure}
\vfill
\newpage

\subsection{¿Cuál es la frecuencia del pico principal?}
Ahora dibujamos en detalle los primeros 20$Hz$, marcamos en rojo los
picos y la escribimos la frecuencia a la que ocurren.
\begin{equation}
f_{PP} = 2.85 Hz
\end{equation}
\vspace{-1cm}
\begin{figure}[h]
\centering
\includegraphics[width = \linewidth]{../plots/fft1.pdf}
\vspace{-1.5cm}
\caption{Detalle de los primeros 20$Hz$ de la \textit{FFT}.}
\label{fig:picos}
\end{figure}

\vspace{-1cm}
\subsection{¿Cuál es el ancho del pico principal?}

Calculamos los picos y las anchuras con la función
\lstinline[style=matlab-editor]{findpeaks} de
\textsc{matlab}.
El pico principal tiene una anchura a media altura (\textit{FWHM}) de
$0.2Hz$ como se muestra en la \hyperref[fig:picos]{\textbf{Fig.}~\ref*{fig:picos}}.
\begin{equation}
FWHM = 0.20 Hz
\end{equation}
\vspace{-1.5cm}
\subsection{¿Tiene  armónicos?  ¿Cuántos? ¿Cómo es la amplitud  de
los armónicos con respecto al pico principal?}

Para investigar los armónicos primero tomamos el vector con las
frecuencias a las que ocurren los picos de la
\hyperref[fig:picos]{\textbf{Fig.}~\ref*{fig:picos}} y dividimos cada
entrada por el valor del segundo pico (el pico principal). Obtenemos
estos resultados:

\begin{table}[h!]
\centering
\begin{tabular}{|r|c|c|c|c|c|c|c|c|c|}
\hline
\textbf{Picos} & \textbf{1} & \textbf{2} & \textbf{3} & \textbf{4} &
	\textbf{5} & \textbf{6} & \textbf{7} & \textbf{8}\\\hline
Frecuencias & 0.47 & 2.85 & 5.94 & 8.43 & 11.28 & 14.01 & 17.57 & 19.59\\\hline
Normalizadas & 0.17 & 1.00 & 2.08 & 2.96 & 3.96 & 4.92 & 6.17 & 6.88\\\hline
Redondeadas  & 0 & 1 & 2 & 3 & 4 & 5 & 6 & 7\\\hline
\end{tabular}
\caption{Frecuencias de los picos.}
\label{table:fcs}
\end{table}
\newpage
En la \hyperref[table:fcs]{\textbf{Tbl.}~\ref*{table:fcs}} vemos que
cuando redondeamos las frecuencias normalizadas al pico principal
obtenemos una secuencia perfecta de números del 1 al 7; esto indica
que los picos se corresponden con los armónicos del segundo pico.
Encontramos que el pico 2 tiene 6 armónicos superiores en los primeros
20Hz de señal, es de esperar que tenga más, pero su amplitud es
demasiado pequeña para poder ser detectados. Las amplitudes se
muestran en al \hyperref[fig:picos]{\textbf{Fig.}~\ref*{fig:picos}}.

\subsection{¿Cuál es la resolución en frecuencia, es decir, el paso entre un punto y otro?}

La resolución en frecuencia indica a partir de qué frecuencia vamos a
ver la señal, y también cada cuánto se van a tomar las muestras.  La
resolución en frecuencia viene dada por el cociente entre la
frecuencia de muestreo y el número de puntos de la transformada de
Fourier, en este caso:
\begin{equation}
f_{r} = \dfrac{250 Hz}{2107} = 0.118652 Hz
\end{equation}

\subsection{¿De  qué  depende  la  resolución  en  frecuencia? ¿Qué  se  podría  hacer  para aumentar el número de puntos, y así aumentar la resolución?}

Vemos que la resolución en frecuencia depende del tramo que escojamos
para el análisis y del envenenado, ya que son las frecuencias que dan
más señal. Si aumentamos el intervalo de muestras o reducimos la
frecuencia de muestreo, la resolución mejora. Si es un tramo demasiado
pequeño no la veremos correctamente, y si es demasiado grande tampoco.
La resolución de compromiso, para estos casos en particular, unos 8
seg puede ser óptimo (aunque también se pueden escoger tramos de  4,
16 segundos...).

\vfill
\begin{figure}[h]
\centering
\includegraphics[width = \linewidth]{../plots/fft2.pdf}
\vspace{-1.5cm}
\caption{Variabilidad del espectro \textit{FFT}.}
\label{fig:LABEL_NAME}
\end{figure}
\vfill
\newpage

\subsection{Calcule la densidad espectral de potencia}

\begin{figure}[h]
\centering
\includegraphics[width = \linewidth]{../plots/welch1.pdf}
\vspace{-1cm}
\caption{Periodograma de \textit{Welch}.}
\label{fig:welch}
\end{figure}


\vfill
\subsection{¿Sobre cuántos tramos se realiza el promediado?}

La señal entera tiene 2107 muestras y nosotros usamos una ventana de
$4 \text{ secs} * f_{s} = 1000$ muestras con un solape de 500 muestras.
El promedio se realizará sobre tres tramos.

\begin{figure}[h]
\centering
\includegraphics[width = \linewidth]{../plots/welch2.pdf}
\vspace{-1cm}
\caption{Periodograma de \textit{Welch}.}
\label{fig:welch}
\end{figure}

\subsection{¿Cómo cambia el ancho del pulso? ¿Por qué?}

La densidad espectral de potencia (PSD) se define como la variación de
energía que hay dentro de una señal vibratoria, en función de la
frecuencia por unidad de masa. Es decir, muestra para cada frecuencia
si la energía presente es mayor o menor.

El periodograma es un estimador de la densidad espectral de potencia
que permite realizar un suavizado del espectro. Con ello obtenemos una
señal más estética pero perdemos resolución. Podemos además escoger
ventanas y realizar solapes entre ellas y así obtener un mayor
suavizado. Se selecciona un segmento, se reduce el tamaño de las
ventanas, se promedia. Además podemos escoger el solape que nos
convenga.

\subsection{¿Qué espectro es más fiable?}

El espectro proporcionado por el periodograma de \textit{Welch} es
menos fiable que el proporcionado por la transformada de
\textit{Fourier}, esto es debido a que el periodograma suaviza la
señal. Este suavizado inherente a la técnica implica
perder parte de la información.

\newpage
\section{Ejercicio 2: Análisis de datos}

En cada registro hemos calculado la posición del pico principal y de
los armónicos con el objetivo de poder caracterizar cada registro y
comparar las distintas arritmias que presentan los pacientes.

\begin{figure}[h]
\centering
\includegraphics[width = \linewidth]{../plots/2welch1.pdf}
\vspace{-1.5cm}
\caption{Periodograma de \textit{Welch} \textbf{001cltv1}.}
\end{figure}

Taquicardia ventricular, se observa un pico principal entre 2-4 Hz, en
concreto en 2.85 Hz. También se ven pequeños armónicos a continuación
del pico principal

\begin{figure}[h]
\centering
\includegraphics[width = \linewidth]{../plots/2welch2.pdf}
\caption{Periodograma de \textit{Welch} \textbf{001cltv4}.}
\end{figure}

Taquicardia ventricular, se observa un pequeño armónico y hay un pico
principal entre los 2 y 4 Hz, concretamente en los 2.81 Hz.
\\\\
\begin{figure}[h]
\centering
\includegraphics[width = \linewidth]{../plots/2welch3.pdf}
\caption{Periodograma de \textit{Welch} \textbf{012clts1}.}
\end{figure}

No desfibrilable, el pico principal está entre 0.8 y 4 Hz, en concreto
en 3.3 Hz. Además presenta armónicos superiores y una gran dispersión.


\begin{figure}[h]
\centering
\includegraphics[width = \linewidth]{../plots/2welch4.pdf}
\caption{Periodograma de \textit{Welch} \textbf{013cltv5}.}
\end{figure}

 Se trata de una taquicardia ventricular ya que presenta armónicos
 pequeños a continuación de éste, aunque el pico principal se
 encuentre en 1.43 Hz (menor a 2 Hz).
\\\\
\begin{figure}[h]
\centering
\includegraphics[width = \linewidth]{../plots/2welch5.pdf}
\caption{Periodograma de \textit{Welch} \textbf{015clsn1}.}
\end{figure}

Es un ritmo sinusal normal, observamos la frecuencia distribuida sin
un claro pico principal

\begin{figure}[h]
\centering
\includegraphics[width = \linewidth]{../plots/2welch6.pdf}
\caption{Periodograma de \textit{Welch} \textbf{022cltv1}.}
\end{figure}

Es una taquicardia ventricular, ya que encontramos el pico principal
está en 2.7 Hz, entre el intervalo de frecuencias 2-4 Hz, y presenta
armónicos pequeños.

\newpage
\section{Ejercicio 3: Base de datos de arritmias ventriculares malignas (MIT)}
\begin{figure}[h]
\centering
\includegraphics[width = \linewidth]{../plots/418.pdf}
\caption{Registro 418.}
\end{figure}

En los primeros 20 segundos encontramos un ritmo sinusal.
Posteriormente, en el intervalo entre 399 - 406 segundos observamos una
fibrilación ventricular, con un pico principal claro y una alta
concentración de frecuencia. En los últimos 20 segundos, se observa un
pico principal aproximadamente en 2 Hz, con algunos armónicos
pequeños, por lo que podríamos decir que se trata de una taquicardia
ventricular, pero no lo podemos asegurar.
\newpage

\begin{figure}[h!]
\centering
\includegraphics[width = \linewidth]{../plots/419.pdf}
\vspace{-1cm}
\caption{Registro 419.}
\end{figure}

Observamos que en los primeros 20 segundos es un ritmo sinusal. En el
intervalo de 320- 340 Hz encontramos una fibrilación auricular, según
lo visto. En el intervalos de 1320-1400 Hz, podemos decir que se trata
de una fibrilación ventricular, ya que observamos un pico principal
con alta concentración y ausencia de armónicos. En los últimos 100
segundos, continúa la fibrilación ventricular y además hay presencia
de ruido en la señal.
\newpage

\begin{figure}[h]
\centering
\includegraphics[width = \linewidth]{../plots/420.pdf}
\caption{Registro 420.}
\end{figure}

En el intervalo que estamos analizando, se encuentra una taquicardia
ventricular y, ya que el pico principal está entre 2-4 Hz y presenta
armónicos pequeños. Además, hay un pico a bajas frecuencias que se
podría deber al ruido.
\\\\

\begin{figure}[h]
\centering
\includegraphics[width = \linewidth]{../plots/421.pdf}
\caption{Registro 421.}
\end{figure}

En la imagen se ve que se trata de una taquicardia ventricular, porque
el pico principal está en el intervalo de 2-4 Hz, con presencia de
pequeños armónicos.

\newpage

\begin{figure}[h]
\centering
\includegraphics[width = 0.8\linewidth]{../plots/422.pdf}
\vspace{-1cm}
\caption{Registro 422.}
\includegraphics[width = \linewidth]{../plots/423.pdf}
\vspace{-2cm}
\caption{Registro 423.}
En los primeros 20 segundos del registro encontramos  un ritmo normal.
En el siguiente intervalo (960-980 segundos) presenta una taquicardia
ventricular. En los últimos 20 segundos, presenta una fibrilación
auricular.
\end{figure}

\newpage
\begin{figure}[h]
\begin{center}
\includegraphics[width = 0.8\linewidth]{../plots/424.pdf}
\end{center}
\vspace{-1cm}
\caption{Registro 424.}
\quad En el intervalo entre 1280-1320 Hz podemos observar que el paciente
presenta una fibrilación ventricular, con un pico principal ancho
entre 4-8 Hz y sin presencia de armónicos.
\includegraphics[width = \linewidth]{../plots/425.pdf}
\vspace{-2cm}
\caption{Registro 425.}
\end{figure}

\newpage

\begin{figure}[h]
\includegraphics[width = \linewidth]{../plots/426.pdf}
\caption{Registro 426.}
\quad Fibrilación Ventricular.
\end{figure}

\begin{figure}[h]
\includegraphics[width = \linewidth]{../plots/427.pdf}
\caption{Registro 427.}
\quad Taquicardia Ventricular
\end{figure}

\end{document}
